\section{Introduction}
\label{sec:introduction}

%Address these questions:
%
%\begin{itemize}
%
%\item What is the context of this research?
%
%\item What is the overall challenge addressed by this research?
%
%\item What is the overall approach adopted by the research?
%
%\item Summarize results very briefly.
%
%\end{itemize}

Revision control systems are a tool most software developers use every day, and of course performing commits is a frequent task in these systems. Each of these commits is associated with a message, so that the reasons for specific changes can be ascertained in the future\cite{CB}. To achieve consistent and useful commit messages, best practices have been established\cite{OffGuide,CB,TP,SR}.

For example, commit messages like ``fixed a bug'' or even just ``bugfix'' serve no practical purpose, since they convey no useful information. Additionally, messages like ``The fnuction x is noww an class y'' and ``makes function x write the output to stdout and changes the format of the output of function x.'' are unpleasant to read in the Git log, because of the changing format, the typos or their verbosity. On the other hand, a message like ``Write output to stdout'' is succinct and tells the reader what it does. Coupled with similarly formatted commit messages, using tools like the Git log is a lot more pleasant.

While there has been research in the field of Mining Software Repositories about emotional content of commit messages (see section \ref{sec:related-work}), none of them have dealt with the actual quality of commits and their relation to the aforementioned best practices. In this paper, we present our approach to this topic.

We have built an application called ``Commit Rater''\footnote{\url{https://github.com/hartenfels/Commit-Rater}}, which extracts commit message data of any Git repository (see section \ref{sec:data-extraction}) and synthesizes the quantitative adherence to commit criteria (see section \ref{sec:data-synthesis}). For the sake of easy use and graphical visualization of the ratings, we have also built a web application\footnote{\url{https://github.com/hartenfels/Commit-Rater-Web}} around the tool, which, at the time of writing, is available for public use at \url{http://trollarena.de:8080/}.

Finally, we also performed machine learning using the Weka data mining software\cite{Weka} to perform association of our commit rating criteria for the sake of improving criteria selection in future versions of the Commit Rater.
