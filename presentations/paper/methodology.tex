\section{Methodology}
% The methodology section must give a very clear account of what sort of
% procedure was executed. Here are some detailed criteria:

Our methodology is divided into the three typical phases of a MSR research project:

% \begin{itemize}
% \item The section is preferably a step-by-step procedure overall. First we did
%   this. Then we did this.
% \item Steps should be motivated or defended, whenever it is not
%   obvious why a certain choice was made. However, one should be
%   careful not to preempt the discussion of threats to validity; see
%   the extra section below.
% \item The text should support reproducibility. In particular, a
%   reasonably knowledgeable person should be able to re-execute the
%   methodology and compare the results.
% \item The text should not get into low-level (implementation)
%   details. These can be deferred to separate documentation; see Appendix~\ref{S:details}.
% \end{itemize}

\begin{itemize}
  \item Data Extraction. Since we wanted to be able to compare arbitrary Git repositories, we decided to \texttt{git clone} repos directly, instead of accessing commits via a proprietary API like GitHub's.
  \item Data Synthesis. This includes checking commits against our criteria, which will be detailed below. The synthesized metrics are counts of met or unmet criteria grouped by the commit author email.
  \item Data Analysis. We use a set of repositories and analyse our synthesis results for interesting statistics and relations between criteria.
\end{itemize}

\subsection{Data extraction}

% This metamodel is concerned with MSR papers. Thus, the methodology
% section would be reasonably subdivided into the three major phases of
% an MSR research project.
%
% The subsection on data extraction is concerned with issues like the
% following:
%
% \begin{itemize}
% \item What repository was chosen and why?
% \item What sort of extraction techniques is applied?
% \item What sort of extra processing (e.g., filtering) is applied?
% \item What constraints were chosen to help with scalability?
% \end{itemize}

% TODO: Describe extraction process
% TODO: Describe preprocessing and output
% TODO: Mention commit-limit
% TODO: Introduce example repos

\subsection{Data synthesis}

Even the simplest MSR project carries out synthesis; data analysis (see
below) may not be carried out in all the cases. The subsection on data
synthesis is concerned with issues like the following:

\begin{itemize}

\item What metrics are used?

\item What machine learning techniques are used?

\item What information retrieval techniques are used?

\end{itemize}

The typical MSR paper picks either metrics or machine learning or IR.

\subsection{Data analysis}

Data analysis would be concerned with any sort of statistical quality
analysis of the data. More specifically, the subsection on data
analysis is concerned with issues like the following:

\begin{itemize}

\item Simple statistics like median of metrics.

\item Analysis of regression or correlation or distribution.

\item Analysis of accuracy such as precision and recall.

\end{itemize}

It should be noted that the methodology section describes the `how'
(and to some extent the`why'), but it does not yet report the various
results; see the following section. For instance, the methodology may
explain why it is using a certain metric and define it and announce
that the median for the metric is going to be determined, as it would
provide a certain insight, but the actual tables or charts for the
metric and the interpretation of the findings would be deferred to the
results section.
