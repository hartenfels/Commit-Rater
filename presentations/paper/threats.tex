\section{Threats to Validity}
\label{sec:threats-to-validity}
%
% There are external and internal threats to validity. See the softlang
% papers and MSR papers for examples. See this resource~\cite{Michael04}
% for an explanation. Be concise and systematic about answering
% questions.

Based on our two research questions, our main conclusions are:
\begin{itemize}
  \item We can recognize a ``good'' commit by a high rating from the Commit Rater since the rating reflects how much the author adhered to certain criteria
  \item Our personally chosen criteria are valid because they show statistical ralations to the criteria taken from official sources
\end{itemize}

\TODO{Revisit this section after drawing real conclusions}

\subsection{Internal Threats to Validity}
\label{sec:internal-threats}

The relation between our criteria and the official criteria may not solely arise from the fact that both are valid criteria. There may be other causes, such as with the relation between the subject line character limit and the subject line word limit, which almost test the same thing.

Furthermore, our tests for the criteria may be flawed. The prime example for such a test is the one for the spelling criterion. As already mentioned in \label{subs:no_misspelling}, spell-checking highly technical texts is very problematic and the results of this test may not be trustworthy.

\subsection{External Threats to Validity}
\label{sec:external-threats}

One basic assumption of our rater is, that the criteria from the official sources define ``good'' commits. This definition may be reasonable in most cases, but some developers might prefer different guidelines. Thus, our rater is only applicable to repositories where these guidelines are agreed upon. Other repositories meight receive significantly worse ratings although the authors may simply have adhered to different guidelines.

Since there is no Ground Truth for if the rated commits are ``good'', we cannot evaluate if our rating method really recognizes such commits. Thus, this assumption may or may not be valid.
