% TODO
% - provide general advice on how to define the research problem
% - ... on how to do practical parts

\documentclass{llncs}

\usepackage{hyperref}
\usepackage{url}
\usepackage{comment}

\pagenumbering{arabic}

\begin{document}

\title{Annotated metamodel for MSR course papers}

\subtitle{Version as of 30 June 2015}

\author{Ralf L\"ammel}

\institute{Software Languages Team, CS Faculty, University of Koblenz-Landau}

\maketitle

\begin{abstract}
  This should be 100-200 words. Use present or past tense, but use it
  consistently. Prefer ``we'' over ``I'' here and throughout the
  paper. The abstract should summarize the methodology and results;
  see the corresponding sections below. Your paper is preferably
  typeset with LaTeX and it should comply with the Springer LNCS
  style. (That style is also available for word users.) 
\end{abstract}

%%%%%%%%%%%%%%%%%%%%%%%%%%%%%%%%%%%%%%%%%%%%%%%%%%%%%%%%%%%%

\section{Introduction}

Address these questions:

\begin{itemize}

\item What is the context of this research?

\item What is the overall challenge addressed by this research?

\item What is the overall approach adopted by the research?

\item Summarize results very briefly.

\end{itemize}

%%%%%%%%%%%%%%%%%%%%%%%%%%%%%%%%%%%%%%%%%%%%%%%%%%%%%%%%%%%%

\section{Research question}

State 1-5 research questions. One question could be enough. More is
not better.

For each question, make clear why this is an interesting question. For
example, try to make clear why a scientist or a practician would
benefit from knowing the answer to the question.

Lay out some background for your research questions. That is, what
body of related work may be worth mentioning and comparing to the
research at hand. This may be related in terms of both relevant
methodology or similar research questions.

You should make sure to carry out a proper related work study. Use
DBLP on the MSR conference for 2+
years.\footnote{\url{http://dblp.uni-trier.de/db/conf/msr/}} You are
welcome to search other conferences and journals. Coming to the
conclusion that there is no related work is absolutely
unacceptable. Again, you need to show related work awareness so that
your research question is substantiated as being relevant and
your methodology is deeply informed by existing publications.

In many cases, you want to refine your research questions into actual
hypotheses, i.e., something that is potentially falsifiable, subject
to an appropriate approach towards measuring.

It is also common to refine research questions. That is, one starts
with a general (overall) research question and takes it appart into
several questions that are concrete enough to be researched on and to
be answered. In this manner, you may often suffice with one or two
research questions which are however refined into perhaps
several questions. Again, even without refinement, just one question may be
Ok. This depends on the nature of the project and on the kind of question.

Here are some recommended MSR papers by the Software Languages
Team~\cite{LaemmelLPV11,LaemmelP13,SchmorleizL15} which you could
consult to see instances of the concepts discussed in this
metamodel. These papers may or may not be in an optimal form to
function as good advice. You are also advised to have a look at some
MSR papers (from the MSR conference), but again, they may also
slightly vary in terms of consistency of style. if you like to get
general advice on scientific writing, then I recommend relevant
textbooks~\cite{Day98,Zobel09}. The best way to learn writing is by
writing and by having someone with experience to provide feedback.

You should most definitely have a look at this excellent tutorial on
empirical software engineering (ESE)~\cite{Easterbrook07} to which MSR
relates. General ESE is quite often concerned with controlled
experiments and other research methods that involve subjects, whereas
MSR is more concerned with automated analysis, but the text is still
very helpful in coming to grips with research methods in software
engineering overall. Specifically, it helps with formulating research questions.

%%%%%%%%%%%%%%%%%%%%%%%%%%%%%%%%%%%%%%%%%%%%%%%%%%%%%%%%%%%%
 
\section{Methodology}

The methodology section must give a very clear account of what sort of
procedure was executed. Here are some detailed criteria:
%
\begin{itemize}
\item The section is preferably a step-by-step procedure overall. First we did
  this. Then we did this.
\item Steps should be motivated or defended, whenever it is not
  obvious why a certain choice was made. However, one should be
  careful not to preempt the discussion of threats to validity; see
  the extra section below.
\item The text should support reproducibility. In particular, a
  reasonably knowledgeable person should be able to re-execute the
  methodology and compare the results.
\item The text should not get into low-level (implementation)
  details. These can be deferred to separate documentation; see Appendix~\ref{S:details}.
\end{itemize}

\subsection{Data extraction}

This metamodel is concerned with MSR papers. Thus, the methodology
section would be reasonably subdivided into the three major phases of
an MSR research project.

The subsection on data extraction is concerned with issues like the
following:

\begin{itemize}
\item What repository was chosen and why?
\item What sort of extraction techniques is applied?
\item What sort of extra processing (e.g., filtering) is applied?
\item What constraints were chosen to help with scalability?
\end{itemize}

\subsection{Data synthesis}

Even the simplest MSR project carries out synthesis; data analysis (see
below) may not be carried out in all the cases. The subsection on data
synthesis is concerned with issues like the following:

\begin{itemize}

\item What metrics are used?

\item What machine learning techniques are used?

\item What information retrieval techniques are used?

\end{itemize}

The typical MSR paper picks either metrics or machine learning or IR.

\subsection{Data analysis}

Data analysis would be concerned with any sort of statistical quality
analysis of the data. More specifically, the subsection on data
analysis is concerned with issues like the following:

\begin{itemize}

\item Simple statistics like median of metrics.

\item Analysis of regression or correlation or distribution.

\item Analysis of accuracy such as precision and recall.

\end{itemize}

It should be noted that the methodology section describes the `how'
(and to some extent the`why'), but it does not yet report the various
results; see the following section. For instance, the methodology may
explain why it is using a certain metric and define it and announce
that the median for the metric is going to be determined, as it would
provide a certain insight, but the actual tables or charts for the
metric and the interpretation of the findings would be deferred to the
results section.

%%%%%%%%%%%%%%%%%%%%%%%%%%%%%%%%%%%%%%%%%%%%%%%%%%%%%%%%%%%%

\section{Results}

\noindent
Summarize results in terms of tables, charts, and other kinds of
figures.

Explain and interpret the results.

Get back to the research question and make sure that it is explicitly answered.

%%%%%%%%%%%%%%%%%%%%%%%%%%%%%%%%%%%%%%%%%%%%%%%%%%%%%%%%%%%%

\section{Threats to validity}

There are external and internal threats to validity. See the softlang
papers and MSR papers for examples. See this resource~\cite{Michael04}
for an explanation. Be concise and systematic about answering
questions.

%%%%%%%%%%%%%%%%%%%%%%%%%%%%%%%%%%%%%%%%%%%%%%%%%%%%%%%%%%%%

\section{Conclusion}

This is essentially a redundant section. Try to summarize your
research from a more informed point of view. (We assume that the
reader has read the paper and is ready for some deeper conclusion.)
You could very well add some perspectives for future work, without
repeating smaller scale weakness of research, as they were already
identified in the threats to validity section.

%%%%%%%%%%%%%%%%%%%%%%%%%%%%%%%%%%%%%%%%%%%%%%%%%%%%%%%%%%%%

\appendix

\section{Details}
\label{S:details}

Any sort of interesting details such as source code, extra data (extra
to the key results), extra procedural details supporting
reproducibility can be listed here. 

%%%%%%%%%%%%%%%%%%%%%%%%%%%%%%%%%%%%%%%%%%%%%%%%%%%%%%%%%%%%

\bibliography{paper}
\bibliographystyle{abbrv}

%%%%%%%%%%%%%%%%%%%%%%%%%%%%%%%%%%%%%%%%%%%%%%%%%%%%%%%%%%%%

\end{document}

%%%%%%%%%%%%%%%%%%%%%%%%%%%%%%%%%%%%%%%%%%%%%%%%%%%%%%%%%%%%
%%%%%%%%%%%%%%%%%%%%%%%%%%%%%%%%%%%%%%%%%%%%%%%%%%%%%%%%%%%%
%%%%%%%%%%%%%%%%%%%%%%%%%%%%%%%%%%%%%%%%%%%%%%%%%%%%%%%%%%%%
%%%%%%%%%%%%%%%%%%%%%%%%%%%%%%%%%%%%%%%%%%%%%%%%%%%%%%%%%%%%
%%%%%%%%%%%%%%%%%%%%%%%%%%%%%%%%%%%%%%%%%%%%%%%%%%%%%%%%%%%%
%%%%%%%%%%%%%%%%%%%%%%%%%%%%%%%%%%%%%%%%%%%%%%%%%%%%%%%%%%%%
%%%%%%%%%%%%%%%%%%%%%%%%%%%%%%%%%%%%%%%%%%%%%%%%%%%%%%%%%%%%
%%%%%%%%%%%%%%%%%%%%%%%%%%%%%%%%%%%%%%%%%%%%%%%%%%%%%%%%%%%%

\section{Introduction}

Generally, this seminar (series) is about ``software languages'' (SL)
and hence it combines elements of software engineering, modelling,
programming languages, linguistics, formal specification, and
programming theory; see a recent discussion of some relevant fields
and terminology~\cite{FavreGLP10}. 

Each particular edition of the SL seminar focuses on a specific topic
within the broader area of SL. (For instance, an edition may focus on
programming-language usage \cite{HageK08} or natural language aspects
in programming \cite{HostO08}.) The present notes provide some general
advice on the reporting and presentation obligations of the seminar's
participants. These notes are meant to be generic with regard to the
particular theme of a specific seminar edition.

That is, the purpose of these notes is to help student participants
writing their reports and preparing their presentations. These notes
complement the meetings and consultations.

\subsection*{Road-map}

In \S\ref{S:report}, advice on the report part of the seminar is
provided.  

\noindent
In \S\ref{S:presentation}, advice on the presentation part
is provided. 

\noindent
These notes are summarized in \S\ref{S:concl}.

\pagebreak

%%%%%%%%%%%%%%%%%%%%%%%%%%%%%%%%%%%%%%%%%%%%%%%%%%%%%%%%%%%%

\section{Advice on the report}
\label{S:report}

Much has been written and said about scientific writing.  A report for
this seminar is a small exercise in scientific writing. Hence,
seminar participants are \emph{strongly encouraged} to gain basic
understanding of scientific writing. For instance, seminar
participants may consult a resource such as
\cite{Zobel09,Day98}. Unfortunately, the syllabus of Koblenz seminars
does not plan for education on scientific writing, neither does any
other obligatory course. Hence, seminar participants are bound to
consult a designated book or online resource, and to work closely
together with the prof to get a reasonable result.\footnote{Here is
  another resource on writing and giving talks:
  \url{http://research.microsoft.com/en-us/um/people/simonpj/papers/giving-a-talk/giving-a-talk.htm}.}
The following notes try to get seminar participants off the ground.

%%%%%%%%%%%%%%%%%%%%%%%%%%%%%%%%%%%%%%%%%%%%%%%%%%%%%%%%%%%%

\subsection{Organization of the report}

As far as reports for the seminar are concerned, we assume a rather
simple organization. A report would be spending about 10-25 pages in
Springer's LNCS style to describe the research work carried out. 

Reports should use close 10 pages if there are not any substantial
illustrations. Reports can use up to 25 pages, but \textbf{a higher
  page count does not imply a better grade}---potentially it may
negatively affect the grade. However, illustrations are strongly
encouraged, if they truly illustrate the subject at hand. So seminar
participants are welcome to exhaust a 25 pages limit with useful
illustrations included.

A report would have a title and an abstract. The purpose of an
abstract is to properly summarize the report: use 50-200 words,
typically 100-150 words.

A report consists of an introduction, several technical sections,
possibly a related work section, and a conclusion section. See below
for more details.

\subsection{The importance of the introduction}

A report should start with an \emph{introduction section} which
motivates the subject of the report, possibly states one or more
\emph{research questions}, typically provides some background material
(if such material is not placed in a subsequent, designated section or
an appendix), lists \emph{contributions} of the report, and describes
the organization of the rest of the report. Judgement of all submitted
work may be biased by the introduction.

If there is any ingredient that is critical to an excellent completion
of the seminar, then this is the formulation of a \emph{research
  question}. By default (for such a short report/paper that is meant
here), the research question should be identified in the introduction
section. The challenge is here to formulate a question that performs
well in all of the following dimensions: a) concise and clear; b)
informed and nontrivial; c) operationally intelligent and verifiable;
d) meaningful to the rest of the text.

\subsection{The technical sections}

After the introduction, there can be any number of \emph{technical
  sections}.  These sections constitute the core (``technical meat'')
of the report. The style of these technical sections very much depends
on the subject and the chosen sources as well as the assumed style of
presentation. 

As there are many different styles, and we do not want to discuss them
here at length. The best advice may be to leverage the corpus of
consulted research papers and to try approximating the style of one of
them. The prof wants to help on this matter---during the consultations
and otherwise.

\subsection{The concluding section}

After the technical sections, the report should be concluded by a
final section. Typically, such a section contains bits of summary,
discussion, interpretations, open problems or future work, and
vision. This may be a good place for the student to show that he or
she has outgrown the view of one or two specific papers and can
usefully operate in a broader research context. Again, an excellent
student would make a serious effort to connect several research papers
and research directions in a comprehensible, informative, and
cross-indexing manner.

\subsection{Literature references}

One important element of scholarly writing is to make appropriate use
of citations. For instance, in the introduction section, one needs to
provide \emph{contextual citations} so that the overall context of the
research effort at hand is clarified. One may also quickly mention
some more technical sources as a means to prepare for the more
detailed parts of the report. Throughout the paper, one uses
\emph{technical citations} for two major purposes: (i) to give credit
to any reused content; (ii) to substantiate claims. In the conclusion
of the report, one uses \emph{visionary citations} to connect to
subjects that were not but cut be covered, or that are worth being
mentioned otherwise.

Seminar participants must prove that they use advanced means of
locating prior work. The \href{http://portal.acm.org/}{ACM Digital
  Library} and the
\href{http://www.informatik.uni-trier.de/~ley/db/}{DBLP} service are
particularly suited to search for scientific publications on topics of
the SL seminar.

Every edition of the SL seminar is unique, but here are some general
pointers for research on software languages; again, every edition of
the SL seminar may draw from additional sources, as it should be
evident from the seminar edition's design. Thus:

\begin{itemize}
\item SLE---Software Language Engineering
\item ASE---Automated Software Engineering
\item ICPC---International Conference on Program Comprehension
\item POPL---Principles of Programming Languages
\item OOPSLA---OO Programming, Systems, Languages, and Applications
\item MODELS--- Model Driven Engineering Languages and Systems
\item WCRE---Working Conference on Reverse Engineering
\item ICSM---International Conference on Software Maintenance
\item MSR---Mining Software Repositories
\item PLDI---Programm Language Design and Implementation
\item PPDP---Principles and Practice of Declarative Programming
\item VL/HCC---Symposium on Visual Languages and Human-Centric Computing
\end{itemize}

There are also various journals on similar themes that may provide
seminal and archival content for the seminar work.

%%%%%%%%%%%%%%%%%%%%%%%%%%%%%%%%%%%%%%%%%%%%%%%%%%%%%%%%%%%%

\subsection{Abstraction is king, not page count}

There is no use in reproducing or condensing one specific paper (or a
series of strongly connected papers). Any reasonable seminar report
must exercise abstraction; it must study a subject from an individual
point of view with an original research question in mind. A reasonable
seminar report involves (3+) different pieces of scholarly work and
integrates them in the presentation.

%%%%%%%%%%%%%%%%%%%%%%%%%%%%%%%%%%%%%%%%%%%%%%%%%%%%%%%%%%%%

\section{Advice on the presentation}
\label{S:presentation}

This section provides another example of what should be called a
technical section. Again, in such sections, the author works out the
technical details (say, meat) of all research work. There can be any
number of such sections. One should think hard about organization
principles to make up the appropriate number of sections and
subsections.

This section briefly talks about the presentation part of the seminar.
Just as in the case of scientific writing, there exists online or
published advice on ``how to give a good talk'', perhaps even in
seminar-like circumstances \cite{Talk,Klaeren94}. Giving good talks is
extremely difficult, but also very rewarding. Speakers should sure to
do some amount of reflection before they go into your talk. Some rules
of thumb follow:

\begin{itemize}

\item Finalizing the slides the night before is perhaps Ok, but one
  should start thinking of the talk way ahead of time. Otherwise, the
  presentation may likely to be boring or stressful and sends the
  audience to sleep. Speakers should not waste the precious cycles of
  the audience! As an aside, the available time should be perfectly
  leveraged by the talk: not more, not less.

\item One should not confuse writing with talking. One should not
  think of the talk as a way to present the paper in a condensed
  style. Instead, one should design the talk in a way that its
  structure and style maintains the attention of the audience and
  keeps up with the audience's abilities. One should use the talk to
  get people interested in the subject. One should motivate the
  audience to read more about the subject in the report that is
  advertised by the presentation.

\item One should not waste time on outlines; they are often totally
  useless at the time they are shown. One should focus on motivation
  from the first slide on. One should make good use of illustrations
  and simplify matters. One should not try to tell every detail, but
  to select some interesting details and make clear what else could
  be discussed or is included into the paper.

\item Seminar talks should be interactive. Hence, clarifying
  questions should be admitted.

\end{itemize}

%%%%%%%%%%%%%%%%%%%%%%%%%%%%%%%%%%%%%%%%%%%%%%%%%%%%%%%%%%%%

\section{Concluding remarks}
\label{S:concl}

These notes are supposed to be useful as is, but further efforts may
be needed to convey all witness and intentions. For the time being,
the author asks the seminar participants and other readers to work
with what is available and the quoted resources. Everyone is welcome
to suggest revisions for these notes.
