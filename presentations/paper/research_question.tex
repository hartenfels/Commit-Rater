\section{Research question}
\label{sec:research-question}

\subsection{Related Work}
\label{sec:related-work}

%State 1-5 research questions. One question could be enough. More is
%not better.
%
%For each question, make clear why this is an interesting question. For
%example, try to make clear why a scientist or a practician would
%benefit from knowing the answer to the question.
%
%Lay out some background for your research questions. That is, what
%body of related work may be worth mentioning and comparing to the
%research at hand. This may be related in terms of both relevant
%methodology or similar research questions.
%
%You should make sure to carry out a proper related work study. Use
%DBLP on the MSR conference for 2+
%years.\footnote{\url{http://dblp.uni-trier.de/db/conf/msr/}} You are
%welcome to search other conferences and journals. Coming to the
%conclusion that there is no related work is absolutely
%unacceptable. Again, you need to show related work awareness so that
%your research question is substantiated as being relevant and
%your methodology is deeply informed by existing publications.
%
%In many cases, you want to refine your research questions into actual
%hypotheses, i.e., something that is potentially falsifiable, subject
%to an appropriate approach towards measuring.
%
%It is also common to refine research questions. That is, one starts
%with a general (overall) research question and takes it appart into
%several questions that are concrete enough to be researched on and to
%be answered. In this manner, you may often suffice with one or two
%research questions which are however refined into perhaps
%several questions. Again, even without refinement, just one question may be
%Ok. This depends on the nature of the project and on the kind of question.
%
%Here are some recommended MSR papers by the Software Languages
%Team~\cite{LaemmelLPV11,LaemmelP13,SchmorleizL15} which you could
%consult to see instances of the concepts discussed in this
%metamodel. These papers may or may not be in an optimal form to
%function as good advice. You are also advised to have a look at some
%MSR papers (from the MSR conference), but again, they may also
%slightly vary in terms of consistency of style. if you like to get
%general advice on scientific writing, then I recommend relevant
%textbooks~\cite{Day98,Zobel09}. The best way to learn writing is by
%writing and by having someone with experience to provide feedback.
%
%You should most definitely have a look at this excellent tutorial on
%empirical software engineering (ESE)~\cite{Easterbrook07} to which MSR
%relates. General ESE is quite often concerned with controlled
%experiments and other research methods that involve subjects, whereas
%MSR is more concerned with automated analysis, but the text is still
%very helpful in coming to grips with research methods in software
%engineering overall. Specifically, it helps with formulating research questions.
